% ======================================================================
\section{The Pixel Unit Cell}
\label{s:puc}
% ======================================================================

The bulk of module testing concerns the pixel unit cell (PUC) of the \roc.  
For this reason, it's useful to discuss the design and operation of the PUC.  
Figure~\ref{fig:puc} shows the elements of the PUC, the double column periphery, and the controller and interface block.  
It is labeled with some of the \dac registers that are most relevant for module testing.  
Terms in black boxes represent programmable \dac registers on the \roc.
While all \dac memory registers are \roc-wide and exist at the periphery (except the 4 \trimbits), 
the diagram shows \dacs next to the circuit element they directly affect.
For the remainder of this document, these \dac registers will be denoted with \textcolor{blue}{blue text}.  
The process of calibrating a module is largely comprised of tests that find and set the optimal values for these parameters.  

\begin{figure*}[hbtp]
\begin{center}
\includegraphics[width=\textwidth]{figures/ROC_dac_schematic.pdf}
\caption{A schematic of the circuit components and the relevant programmable registers within the \roc.}
\label{fig:puc}
\end{center}
\end{figure*}

The signal processing within the PUC begins with an amplifier and shaper that convert the incident current into a reformatted voltage signal.  
The signal is then fed into a voltage comparator, 
the threshold of which is controlled by \vthrcomp with additional inputs possible from \vtrim and the \trimbits.  
If the input signal exceeds the comparator's threshold, 
the signal is stored for readout and the pixel notifies the logic at the periphery of the double column 
that a hit has been registered and is ready to be transferred to the data buffer at the periphery.   
This initiates the double column drain mechanism, 
transferring pulse height (PH) values from any pixel with a hit registered since the previous drain onto the 80-wide PH buffer on the periphery.
In addition to the ability to process charge produced by particles passing through the silicon sensor, 
the \roc can produce an internal calibration signal whose amplitude is controlled via the \vcal~\dac.  
This calibration signal serves as the input to the PUC in almost all of the tests described in this document.
Table~\ref{tab:daclist} shows a subset of \dac registers in the PUC and explains their usage.
\\\\

\begin{table}[htbp]
\caption{List of \dac registers relevant to module testing.  \dac names are denoted with \textcolor{blue}{blue} text.}
\renewcommand{\arraystretch}{1.2}\begin{tabular}{|c|L|L|}
\hline
\dac Name & Usage & Tests that optimize this \dac \\
\hline
\hline
\vcal & Calibration signal voltage. There are two \vcal ranges (low and high) with the high range being roughly 7 times larger.  
The choice of low/high range is set via the \textcolor{blue}{CtrlReg} \dac. & None - set to an appropriate level depending on the test\\
\hline
\vana & Voltage supplied to the analog part of the PUC (amplifier and shaper) & \pretest ({\tt SetVana}) \\
\hline
\caldel & Delay in sending out calibration pulse with respect to the clock signal & \pretest ({\tt SetVthrCompCalDel}) \\
\hline
\vthrcomp & Voltage supplied to the comparator unit, inversely related to the signal voltage required to fire the comparator (i.e. the comparator threshold) & \pretest ({\tt SetVthrCompCalDel}), \trimming \\
\hline
\vtrim & Additional voltage supplied to the comparator unit to decrease the threshold (in conjunction with the \trimbits) & \trimming \\
\hline
\trimbits & 4 bits per PUC that decrease the threshold of the comparator unit & \trimming \\
\hline
\phoffset & Constant offset added to the pulse height & \phopt \\
\hline
\phscale & Sets the slope of the PH vs. input signal strength curve & \phopt \\
\hline
\end{tabular}
\label{tab:daclist}
\end{table}
