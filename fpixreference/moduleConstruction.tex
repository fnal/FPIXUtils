% ======================================================================
\section{FPix Module Construction}
\label{s:contruction}
% ======================================================================

Module assembly takes place in several independent steps at multiple locations.  
First, the sensor is attached to 16 ROCs by the company RTI using a bump-bonding technique.  
Balls of lead/tin solder, roughly 30-40 $\mu$m in diameter, are deposited onto the ROCs.  
Then each ROC is aligned to the sensor and the two are pressed together to complete the electrical connection.

HDIs, produced by Compunetics, are delivered to Fermilab where all the surface components (including the TBM) are mounted.

The combination of 16 ROCs bump-bonded to a sensor is referred to as a “bare module”.  
RTI sends the bare modules to the two FPix assembly site at Purdue University and The University of Nebraska.  
At the assembly sites, the completed HDIs are glued to the top of the sensor side of the bare modules.  
Then the readout pads of the ROCs are connected to the input pads of the HDI via thin wires in a process referred to as wirebonding.  
The complete modules are tested for basic functionality, and any correctable errors (e.g. broken wirebonds) are addressed.

The modules are then shipped to one of three FPix testing sites:  
Fermilab (FNAL), University of Kansas (KU), and University of Illinois-Chicago (UIC).  
KU and UIC are equipped with X-ray testing setups, 
while FNAL is the central testing site where the final calibration and grading is performed.  
This document will detail the testing procedures at FNAL, as X-ray testing is covered in a separate document.
